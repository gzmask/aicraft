\title{The A.I Craft Project}
\author{
	Ray Shulang Lei\\
	200253624\\
	ray.lei@uregina.ca
}
\date{\today}

\documentclass[12pt]{article}
\setlength{\parindent}{0in}
\usepackage{graphicx}
\usepackage{mathtools}
\usepackage{amsthm}
\usepackage{parskip}

\begin{document}
\maketitle

\begin{abstract}
This is the proposed paper of a multiplayer cloud base livecoding A.I combating game, similar to robocode, and plan to be implemented with NodeJS and WebGL.
\end{abstract}

\section{Introduction}
A.I craft is an online A.I combating game. Players use their livecoding skills to provide code in real time to control their A.I crafts, in order to beat the opposite players. The idea is inspired by the JAVA Robocode\cite{robocode01} project.

\section{The Funativity reasonale behind the A.I Craft concept}
In Noah Falstein's article "Natural Funativity"\cite{noah2004}, he described three types of hunters in a hunter-gatherer society. After all hunters have got enough haunch of antelope, enough to feed his family for a few days:\\
Aagh - goes right back out to track down a deer he saw earlier;\\
Bohg - passes his time by kicking back and catching nothing more challenging than some rays;\\
Cragh - plays hunting games at home. "Like Aagh, he is building survival skills - but does so it in the safe confines of home. Like Bohg, he stays safe - but also stays fit and slightly improves his chances for success in the next hunt. "\\
Noah Falstein concluded that "Over hundreds of thousands of generations, those genes that Cragh carries are more likely to spread, and the activities - including games - can also be passed on through word of mouth in the tribe from generation to generation."\\

According to Venkatesh's article "The Rise of Developeronomics"\cite{venkatesh2011} on Forbes, our society now is a developer-centric economy. Thus there are only two kinds of people matters: developers, and people who controls developers. Assuming our society is developer driven, I can conclude that the developers are as important as hunters in a hunter-gatherer society. Now what Aagh, Bohg and Cragh would be like in today's society? After developers finishes their work before the deadline, there would be three types of behaviors:
Aagh - goes right back out to coding for more incoming, whether it is a contract job or freelancing;\\
Bohg - chilling out, get more sleep, let the brain rest;\\
Cragh - plays games at home to sharpen his development skills.\\ 

Now, let us wait a second and rethink about what kind of games Cragh can use to improve his development skills. First person shooters? It only makes a better hunter. Strategy games? Maybe, but not exactly. Thus we need a new genre of games: coding games. If Noah and Venkatesh are right, developers who plays coding games are more likely to spread their genes. And right here I have found the goal of making A.I Craft: to imporve the chance of spreading my genes... So it makes perfect sense to put my best effort fulfilling it.

\section{Design}


\begin{thebibliography}{9}

\bibitem{robocode01}
	Mathew Nelson, Flemming Larsen, IBM etc.\\
	\emph{Robocode}.\\
	http://robocode.sourceforge.net\\
	2001

\bibitem{noah2004}
	Noah Falstein,\\
	\emph{Natural Funativity}.\\
	http://www.gamasutra.com/view/feature/2160/natural\_funativity.php\\
	2004.

\bibitem{venkatesh2011}
	Venkatesh Rao,\\
	\emph{The Rise of Developeronomics}.\\
	http://www.forbes.com/sites/venkateshrao/2011/12/05/the-rise-of-developeronomics/\\
	2011.

\end{thebibliography}

	

\end{document}

